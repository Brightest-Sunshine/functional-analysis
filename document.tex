\documentclass{article}
\usepackage[utf8]{inputenc}
\usepackage[english,russian]{babel}
\usepackage{amsmath}
\usepackage{tikz,amstext}
\usepackage{graphicx}
\usepackage{amssymb}
\usepackage{lscape}
\usepackage{indentfirst}
\newlength{\tempheight}
\graphicspath{{/.graph/}}

\newcommand{\Let}[0]{%
	\mathbin{\text{\settoheight{\tempheight}{\mathstrut}\raisebox{0.5\pgflinewidth}{%
				\tikz[baseline,line cap=round,line join=round] \draw (0,0) --++ (0.4em,0) --++ (0,1.5ex) --++ (-0.4em,0);%
}}}}


\begin{document}


	\section{Глава}
		\paragraph{\S 1 Операции над множествами. Крякря}\par
		
			\hfill\par
			 $\Let A, B;$ $A\in X$; $x,y\in A$;$\{A_\alpha\}_{\alpha\in\Lambda}$
			\begin{equation*}
				\begin{aligned}
					\bigcup_{\alpha\in\Lambda} A_\alpha &= \{x\in X \bigm| \exists \alpha_0\in\Lambda : x\in A_{\alpha_0 } \}\\
					\bigcap_{\alpha \in \Lambda} A_\alpha &= \{x\in X \bigm | x \in A_\alpha; \forall \alpha \in \Lambda \}\\
					A\setminus B &= \{x \in X \bigm | x\in A; X \notin B\}\\
					A^\mathrm{C}=cA&=\{X\notin A = x \in X \bigm | x \notin A\}
				\end{aligned}
			\end{equation*}
			
		

	
		
			 
			 
			 \textbf{Утв. 1} {\it 
			 	\begin{equation*}
			 		\begin{aligned}
			 			c\left(\bigcup_{\alpha\in\Lambda} A_\alpha\right) = \bigcap_{\alpha \in \Lambda} \left(cA_\alpha\right)
			 		\end{aligned}
			 	\end{equation*}\par.
			 }
			 \\
			 \textbf{Доказательство.}
			 \begin{equation*}
			 \begin{aligned}
				\Let  x \in c\left( \bigcup_{\alpha\in\Lambda} A_\alpha\right) \Leftrightarrow x\notin \bigcup_{\alpha\in\Lambda} A_\alpha \Leftrightarrow \nexists \alpha_0 \in \Lambda : x\in A_{\alpha_0}
				\Leftrightarrow x\notin A_\alpha,\forall \alpha \in \Lambda
				\Leftrightarrow x \in cA_\alpha,\forall\alpha\in\Lambda
				\Leftrightarrow x\in\bigcap_{\alpha \in \Lambda} \left(cA_\alpha\right)
				\Box
			 \end{aligned}
			 \end{equation*}\par

			  \textbf{Утв. 2} {\it 
			  	\begin{equation*}
			  		\begin{aligned}
			  			c\left(\bigcap_{\alpha\in\Lambda} A_\alpha\right) = \bigcup_{\alpha \in \Lambda} \left(cA_\alpha\right)
			  		\end{aligned}
			  	\end{equation*}\par.}
			 \\
			 \textbf{Доказательство.}
			 \begin{equation*}
			 	\begin{aligned}
			 		&\Let x \in c\left(\bigcap_{\alpha\in\Lambda} A_\alpha\right) \Leftrightarrow
			 		x \notin \bigcap_{\alpha\in\Lambda} A_\alpha \Leftrightarrow
			 		\exists \alpha_0 \in\Lambda : x \in A_{\alpha_0} \Leftrightarrow
			 		x \in cA_{\alpha_0} \Leftrightarrow
			 		x \in \bigcup_{\alpha \in \Lambda} \left(cA_\alpha\right)
			 		\\
			 		&\text{. Поскольку }
			 		\exists cA_{\alpha_0} : x\in  cA_{\alpha_0}
					\Box
				 \end{aligned}			 
			 \end{equation*}\par
			
			
	
\end{document}
